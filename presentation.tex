\documentclass[%
    aspectratio=169, % Aspect ratio. 43 and 169 are supported for the title slide
    onlytextwidth,   % Restricts the beamer columns environment to only use the text width
]{beamer}
\usepackage{basileabeam}

% Notes:
%\pgfpagesuselayout{2 on 1}[a4paper,border shrink=5mm]
%\setbeamertemplate{note page}[plain]
%\setbeameroption{show notes on second screen=bottom}

\title              {Title}

\author             {Author}
\email              {author@email.com}
\institute          {Institute, University of Basel}

\date               {Date}

\ulogo        		{Template/header}
\ulistelement    	{Template/listelement}

\graphicspath{{Figures/}}

% Options:
\totalNoSlidesDisabled % To turn off the total number of slides in the footer. Comment this if you want the total number of slides in the footer

\headerSectionsDisabled % Comment this if you want a fancy header containing your sections.

% have the University of Basel logo on the title slide
\unibasLogoEnabled

% Use frame numbers in contrast to page numbering, or comment it to have page numbers again
\pageNumberingEnabled

\begin{document}

\begin{frame}[t,plain]
\titlepage
\end{frame}

\note{Notes can help you to remember important information. Turn on the notes option.}

\section{Section 1}	% You can also have slides prior to the first section or work entirely without sections.

\begin{frame}[label=turing,c]{Some Images}
\begin{columns}[c]
    \column{.55\textwidth}
            \includegraphics[width=0.8\textwidth]{block}
    \column{.45\textwidth}
            Turing machine \cite{turing:1936,turing:1950}
\end{columns}
\end{frame}

\note{Notes can help you to remember important information. Turn on the notes option.}

\begin{frame}[c]{Some Images}
    \begin{figure}
        \includegraphics[width=0.8\textwidth]{turingmachine}
        \caption{A Turing Machine.}
    \end{figure}
\end{frame}

\note{Notes can help you to remember important information. Turn on the notes option.}

\begin{frame}[c]{Some Equations}
Now we introduce an equation.
\begin{theorem}
A Turing Machine is a 7-Tuple:
\begin{equation}
    M = \langle Q, \Gamma, b, \Sigma, \delta, q_0, F \rangle
\end{equation}
\end{theorem}
A Turing Machine is a 7-Tuple even if defined in the text, as in $M = \langle Q, \Gamma, b, \Sigma, \delta, q_0, F \rangle$.
\end{frame}

\begin{frame}[c]{Beamer Overlays}
    \begin{columns}
        \column<+->{.3\textwidth}
            This page consists of three columns,
        \column<+->{.3\textwidth}
            which are gradually revealed
        \column<+->{.3\textwidth}
            with the power of Beamer Overlays.
    \end{columns}
    \bigskip{}

    Check out the beamer class user documentation \url{https://texdoc.org/serve/beamer/0}, in particular chapter 9 for overlays.
    \bigskip{}
    
    This slide also demonstrates that one should add the beamer option \texttt{onlytextwidth}.
\end{frame}

\note{Notes can help you to remember important information. Turn on the notes option.}

\section{Section 2}

\begin{frame}[t]{Items and Numbers}
\begin{columns}
    \column{.5\textwidth}
            \begin{itemize}
            \item one
            \item two
            \item three
            \end{itemize}
    \column{.5\textwidth}
            \begin{enumerate}
            \item first
            \item second
            \item third
            \end{enumerate}
\end{columns}
\end{frame}


\note{Notes can help you to remember important information. Turn on the notes option.}

\begin{frame}[c]{Tables}
Tables are also interesting.
\begin{table}[ht!]
\centering
\begin{tabular}{|l|c|l|} \hline
Title&$f$&Comments\\ \hline
The chemical basis of morphogenesis & 7327 & \\ \hline
On computable numbers & 6347 & Turing Machine\\ \hline
Computing machinery and intelligence & 6130 & \\ \hline
\end{tabular}
\end{table}
\end{frame}

\note{Notes can help you to remember important information. Turn on the notes option.}

\againframe{turing} % Search this file for label=turing to find the frame definition

\begin{frame}[c]{Speaker Notes}
You may turn on the notes and handout option to see the notes to the slides.
\end{frame}

\note{Notes can help you to remember important information. Turn on the notes option.}

\begin{frame}[t,plain]
\lastpage{{\usebeamerfont{title} Questions?}\\[5ex]
author@email.com}
\end{frame}

\note{Notes can help you to remember important information. Turn on the notes option.}

\backupbegin

\begin{frame}{Backup}
Test
\end{frame}

\backupend

\end{document}

