\documentclass{beamer}
\usepackage{basilea}

\title              {Title}

\author             {Author}
\email              {author@email.com}
\institute          {Institute, University of Basel}

\date               {Date}

\universitylogo		{Template/header}
\universitysignet	{Template/signet}



\begin{document}
\begin{frame}[t,plain]
\titlepage
\end{frame}

\note{Notes can help you to remember important information. Turn on the notes option.}

\begin{frame}[c]{Some Images}
\begin{columns}
    \column{.55\textwidth}
            \includegraphics[width=0.8\textwidth]{block}
    \column{.45\textwidth}
            \begin{itemize}
            \item Turing machine 
            \end{itemize}
\end{columns}
\end{frame}

\note{Notes can help you to remember important information. Turn on the notes option.}

\begin{frame}[c]{Some Images}
    \begin{figure}
        \includegraphics[width=0.8\textwidth]{turingmachine}
    \end{figure}
\end{frame}

\note{Notes can help you to remember important information. Turn on the notes option.}

\begin{frame}[c]{Some Equations}
Noew we introduce an equation.
\begin{theorem}
A Turing Machine is a 7-Tuple:
\begin{equation}
    M = \langle Q, \Gamma, b, \Sigma, \delta, q_0, F \rangle
\end{equation}
\end{theorem}
A Turing Machine is a 7-Tuple even if defined in the text, as in $M = \langle Q, \Gamma, b, \Sigma, \delta, q_0, F \rangle$.
\end{frame}

\note{Notes can help you to remember important information. Turn on the notes option.}

\begin{frame}[c]{Tables}
Tables are also interesting.
\begin{table}[ht!]
\centering
\begin{tabular}{|l|c|l|} \hline
Title&$f$&Comments\\ \hline
The chemical basis of morphogenesis & 7327 & \\ \hline
On computable numbers & 6347 & Turing Machine\\ \hline
Computing machinery and intelligence & 6130 & \\ \hline
\end{tabular}
\end{table}
\end{frame}

\note{Notes can help you to remember important information. Turn on the notes option.}

\begin{frame}[c]{Speaker Notes}
You may turn on the notes and handout option to see the notes to the slides.
\end{frame}

\note{Notes can help you to remember important information. Turn on the notes option.}

\begin{frame}[t,plain]
\lastpage{Questions? Comments?}
\end{frame}

\note{Notes can help you to remember important information. Turn on the notes option.}

\end{document}

